\documentclass[12pt,a4paper]{moderncv}
\moderncvtheme[blue]{classic} %[blue, green, orange, red, grey]{casual, classic}
%\definecolor{color1}{RGB}{138,74,57}
\usepackage{lmodern}
\usepackage[ngerman]{babel}
\usepackage{amsmath}
\usepackage[utf8]{inputenc}
\firstname{Tim Owe}
\familyname{Wisch}
\title{Lebenslauf}
\address{Heider Str. 15}{24106 Kiel}
\mobile{0160 / 966 83 963}
\email{owe.wisch@gmail.com }
\homepage{github.com/OweWi}
\photo[3cm]{img/owe}
\begin{document}
\maketitle

\cvline{}{}{}
\renewcommand*{\cventry}[6]{%
  \cvline{#1}{%
    {\bfseries#2}%
    \ifx#3\else{, {\slshape#3}}\fi%
    \ifx#4\else{, #4}\fi%
    \ifx#5\else{, #5}\fi%
    \ifx#6\else{\newline{}\begin{minipage}[t]{\linewidth}\small#6\end{minipage}}\fi
    }}%

%%
\section{Persönliche Angaben}
\cventry{Geburtstag}{30. Dezember 1991}{Heide}{}{}{}
\cventry{Familienstand}{verheiratet}{}{}{}{}
\cvline{}{}{}

%%
\section{Ausbildung}
\cvline{}{}{}
\subsection{Akademische Ausbildung und Berufserfahrung}
\cventry{11/2017 -- heute}{Wissenschaftlicher Mitarbeiter}{\newline Lehrstuhl für digitale Signalverarbeitung und Systemtheorie}{Schwerpunkt: Unterwasser-Echtzeitsignalverarbeitung und Unterwasserkommunikation}{}{}
\cventry{04/2017--10/2017}{Masterarbeit}{\newline \textit{Spracherkennung in stark gestörten Unterwasserumgebungen unter Nutzung von wasserfesten Mikrofonen} }{}{}{}
\cventry{10/2015--03/2017}{wiss. Hilfskraft}{\newline Lehrstuhl für digitale Signalverarbeitung und Systemtheorie}{}{}{}
\cventry{2015--2017}{Elektro- und Informationstechnik}{Christian-Albrechts-Universität zu Kiel}{ \newline Abschluss: Master of Science}{}{}
\cventry{2015--2017}{Wirtschaftsingenieurwesen Elektro- und Informationstechnik}{Christian-Albrechts-Universität zu Kiel}{\newline Abschluss: Master of Science}{}{}{}
\cventry{04/2015--08/2015}{Bachelorarbeit}{\newline \textit{Implementierung eines Filtered-X LMS-Algorithmus
zur aktiven Störungsunterdrückung auf einer
DSP-Plattform}}{}{}{}
\cventry{09/2014--03/2015}{Praktikum}{Volkswagen AG}{Wolfsburg}{\newline Elektronikanalyse auf Hardwareebene und Kostenkalkulation}{}{}
\cventry{03/2014--07/2014}{stud. Hilfskraft}{\newline Lehrstuhl für digitale Signalverarbeitung und Systemtheorie}{}{}{}
\cventry{2011--2015}{Elektro- und Informationstechnik}{Christian-Albrechts-Universität zu Kiel}{\newline Abschluss: Bachelor of Science}{}{}
\cventry{2011--2015}{Wirtschaftsingenieurwesen Elektro- und Informationstechnik}{Christian-Albrechts-Universität zu Kiel}{\newline Abschluss: Bachelor of Science}{}{}
\cventry{08/2011--09/2011}{Vorpraktikum}{Maschinenbau Lorenzen, Husum}{\newline Mechanische Grundpraxis Metallbearbeitung}{}{}
\subsection{Schule}
\cventry{07/2011}{Allgemeine Hochschulreife}{}{}{}{}
\cventry{2002--2011}{Gymnasium}{Werner-Heisenberg-Gymnasium}{Heide}{}{}
\cvline{}{}{}
%%
\section{Qualifikationen}
\cvline{}{}{}

\subsection{Sprachkenntnisse}
\cvlanguage{Deutsch}{Muttersprache}{}{}
\cvlanguage{Englisch}{fließend}{}{}
\cvline{}{}{}

\subsection{EDV Kenntnisse}
\cventry{Programmiersprachen}{}{}{}{}{}
\cventry{}{C/C++}{}{}{}{}
\cventry{}{Python}{}{}{}{}
\cventry{}{MATLAB}{}{}{}{}
\cventry{Frameworks}{}{}{}{}{}
\cventry{}{Qt}{C++}{}{}{}
\cventry{}{Numpy}{Python}{}{}{}
\cventry{}{Tensorflow}{Python}{}{}{}
\cventry{Office}{}{}{}{}{}
\cventry{}{Word}{}{}{}{}
\cventry{}{Excel}{}{}{}{}
\cventry{}{Powerpoint}{}{}{}{}
\cventry{}{CorelDraw}{}{}{}{}
\cventry{}{\LaTeX}{}{}{}{}
\cventry{CAD}{}{}{}{}{}
\cventry{}{Autodesk Fusion 360}{}{}{}{}
\cventry{}{KiCAD}{}{}{}{}

% \newpage
%%
\section{Sonstiges}
\cventry{Patente}{}{}{}{}{}
\cventry{}{(Mit-)Erfinder}{DE102021112535B3}{SONAR-Verfahren zur Erkennung und/oder Positions- und/oder Geschwindigkeitsbestimmung von Objekten unter Wasser in einem vorbestimmten Gebiet sowie eine SONAR-Anordnung und Empfangseinheit}{Erteilung 07.04.2022}{}
\cventry{Lehrveranstaltungen}{}{}{}{}{}
\cventry{}{Digitale Signalverarbeitung}{Übung}{Bachelor}{}{}
\cventry{}{Advanced Digital Signal Processing}{Übung}{Master}{}{}
\cventry{}{System Theory Lab}{Labor/Praktikum}{Bachelor}{MATLAB-Praktikum mit Anwendungen in der Signalverarbeitung}{}
\cventry{}{Machine Learning Lab}{Labor/Praktikum}{Master}{Versuch über Autoencoder mit Tensorflow}{}

\cventry{Mitgliedschaften}{}{}{}{}{}
\cventry{}{VDE}{studentisches Mitglied}{}{}{}
\cventry{}{Betriebssportverband Kiel}{Spieler, Schiedsrichter}{Fußball}{}{}
\cventry{}{Gettorfer SC Sportclub von 1948}{Spieler}{Fußball}{}{}
\cvline{}{}{}
\cvline{}{}{}



Kiel, \today
\end{document}
